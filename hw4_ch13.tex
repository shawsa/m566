\documentclass[12pt]{article}

%%%%%%%%%%%%%%%%%%%%%%%%%%%%%%%%%%%%%%%%%%%%%%%%%%%%%%%%%%%%%%%%%%%%%%%%%%%%%%%%%%%%%%%%%%%%%%%%%%%%
% Math
\usepackage{fancyhdr} 
\usepackage{amsfonts}
\usepackage{amsmath}
\usepackage{amssymb}
\usepackage{amsthm}
\usepackage{enumitem}
%\usepackage{dsfont}

%%%%%%%%%%%%%%%%%%%%%%%%%%%%%%%%%%%%%%%%%%%%%%%%%%%%%%%%%%%%%%%%%%%%%%%%%%%%%%%%%%%%%%%%%%%%%%%%%%%%
% Macros
\usepackage{calc}

%%%%%%%%%%%%%%%%%%%%%%%%%%%%%%%%%%%%%%%%%%%%%%%%%%%%%%%%%%%%%%%%%%%%%%%%%%%%%%%%%%%%%%%%%%%%%%%%%%%%
% Commands and Custom Variables	
\newcommand{\problem}[1]{\hspace{-4 ex} \large \textbf{Problem #1} }
\let\oldemptyset\emptyset
\let\emptyset\varnothing
\newcommand{\norm}[1]{\left\lVert#1\right\rVert}
\newcommand{\sint}{\text{s}\kern-5pt\int}
\newcommand{\powerset}{\mathcal{P}}
\renewenvironment{proof}{\hspace{-4 ex} \emph{Proof}:}{\qed}
\newcommand{\RR}{\mathbb{R}}
\newcommand{\NN}{\mathbb{N}}
\newcommand{\QQ}{\mathbb{Q}}
\newcommand{\ZZ}{\mathbb{Z}}
\newcommand{\CC}{\mathbb{C}}
\renewcommand{\Re}{\operatorname{Re}}
\renewcommand{\Im}{\operatorname{Im}}

\newcommand{\solution}{\vspace{2 ex} \hspace{-5 ex} \emph{Solution.} }


%%%%%%%%%%%%%%%%%%%%%%%%%%%%%%%%%%%%%%%%%%%%%%%%%%%%%%%%%%%%%%%%%%%%%%%%%%%%%%%%%%%%%%%%%%%%%%%%%%%%
%page
\usepackage[margin=1in]{geometry}
\usepackage{setspace}
%\doublespacing
\allowdisplaybreaks
\pagestyle{fancy}
\fancyhf{}
\rhead{Malmuth \& Shaw \space \thepage}
\setlength\parindent{0pt}

%%%%%%%%%%%%%%%%%%%%%%%%%%%%%%%%%%%%%%%%%%%%%%%%%%%%%%%%%%%%%%%%%%%%%%%%%%%%%%%%%%%%%%%%%%%%%%%%%%%%
%Code
\usepackage{listings}
\usepackage{courier}
\lstset{
	language=Python,
	showstringspaces=false,
	formfeed=newpage,
	tabsize=4,
	commentstyle=\itshape,
	basicstyle=\ttfamily,
}

%%%%%%%%%%%%%%%%%%%%%%%%%%%%%%%%%%%%%%%%%%%%%%%%%%%%%%%%%%%%%%%%%%%%%%%%%%%%%%%%%%%%%%%%%%%%%%%%%%%%
%Images
\usepackage{graphicx}
\graphicspath{ {images/} }
\usepackage{float}

%tikz
\usepackage[utf8]{inputenc}
\usepackage{pgfplots}
\usepgfplotslibrary{groupplots}

%%%%%%%%%%%%%%%%%%%%%%%%%%%%%%%%%%%%%%%%%%%%%%%%%%%%%%%%%%%%%%%%%%%%%%%%%%%%%%%%%%%%%%%%%%%%%%%%%%%%
%Hyperlinks
%\usepackage{hyperref}
%\hypersetup{
%	colorlinks=true,
%	linkcolor=blue,
%	filecolor=magenta,      
%	urlcolor=cyan,
%}

\begin{document}
	\thispagestyle{empty}
	
	\begin{flushright}
		Daniel Malmuth \& Sage Shaw \\
		m566 - Spring 2018 \\
		\today
	\end{flushright}
	
{\large \textbf{HW - Chapter 13}}\bigbreak

\problem{5} The code listed in the appendix does as this problem asks. When ran, it gives an output of $3.78285*10^{-15}$ telling us that the left hand side and the right hand side of the identities are equal to within machine precision. 

\bigbreak
\problem{10} Let $m,n,l \in \ZZ^+$ such that $m=n+1=2l$. Let $x_\alpha = \frac{2\pi \alpha}{m}$ for $\alpha=0,1,...,m-1$. From section 13.3 we are given that for $-l \leq j \leq l-1$
\begin{align*}
	c_j & = \frac{1}{m} \sum\limits_{\alpha=0}^{m-1} y_\alpha e^{-ijx_\alpha} \\
	& = \frac{1}{m} \sum\limits_{\alpha=0}^{m-1} y_\alpha \left( \cos(jx_\alpha) - i\sin(jx_\alpha)\right)
\end{align*}
From section 13.2 we are given that
\begin{align*}
	a_k &= \frac{1}{l} \sum\limits_{\alpha=0}^{n} y_\alpha \cos(k x_\alpha) & \text{for }k&=0, 1, ..., l \\
	b_k &= \frac{1}{l} \sum\limits_{\alpha=0}^{n} y_\alpha \sin(k x_\alpha) & \text{for }k&=0, 1, ..., l-1 \\
\end{align*}
Note that $lx_\alpha = l \frac{2\pi \alpha}{m} = \alpha \pi$ and thus $\sin(l x_\alpha) = 0$ for any integer $\alpha$. We will define $b_l=0$ and notice that this definition extends the formula above to $k=0,...,l$. \\
Suppose that $0 \leq j \leq l-1$. Also note that $\frac{1}{l} = \frac{2}{m}$. Then
\begin{align*}
	a_j - ib_j & = \frac{1}{l} \sum\limits_{\alpha=0}^{n} y_\alpha \cos(j x_\alpha) - i \frac{1}{l}\sum\limits_{\alpha=0}^{n} y_\alpha \sin(j x_\alpha) \\
	& = \frac{1}{l} \sum\limits_{\alpha=0}^{n} y_\alpha \left( \cos(j x_\alpha) - i\sin(j x_\alpha) \right) \\
	& = \frac{2}{m} \sum\limits_{\alpha=0}^{n} y_\alpha e^{-ijx_\alpha} \\
	& = c_j
\end{align*}
Suppose that $-l \leq j < 0$. Then
\begin{align*}
a_{-j} + ib_{-j} & = \frac{1}{l} \sum\limits_{\alpha=0}^{n} y_\alpha \cos(-j x_\alpha) + i \frac{1}{l}\sum\limits_{\alpha=0}^{n} y_\alpha \sin(-j x_\alpha) \\
& = \frac{1}{l} \sum\limits_{\alpha=0}^{n} y_\alpha \left( \cos(-j x_\alpha) + i\sin(-j x_\alpha) \right) \\
& = \frac{1}{l} \sum\limits_{\alpha=0}^{n} y_\alpha \left( \cos(j x_\alpha) - i\sin(j x_\alpha) \right) \\
& = \frac{2}{m} \sum\limits_{\alpha=0}^{n} y_\alpha e^{-ijx_\alpha} \\
& = c_j
\end{align*}
Thus we have the relationship 
\begin{align*}
	c_j &= \begin{cases}
		a_j - ib_j & 0\leq j \leq l-1 \\
		a_{-j} + ib_{-j} & -l+1 \leq j < 0 \\
		a_{l} & j=-l
	\end{cases}
\end{align*}
From these, it is easy to derive the reverse relationship
\begin{align*}
	a_j &= \begin{cases}
		\frac{c_j + c_{-j}}{2} & 0 \leq j \leq l-1 \\
		c_{-l} & j=-l
	\end{cases} \\
	b_j &= \frac{c_j - c_{-j}}{2i}
\end{align*}

{\hspace{-4 ex} \large \textbf{Appendix - Code listings}}\bigbreak

\problem{5} Code for problem 5:
\begin{lstlisting}
from math import sin, cos, pi

def left_hand_cos(m, j, k):
	xs = [2*pi*i/m for i in range(m)]
	ret = sum( [cos(k*xi)*cos(j*xi) for xi in xs] )
	return 2/m * ret

def right_hand_cos(m, j, k):
	if k!=j:
		return 0
	elif 0<k<m/2 and 0<j<m/2:
		return 1
	elif (k==0 and j==0) or (j==m/2 and k==m/2):
		return 2	

def left_hand_sin(m, j, k):
	xs = [2*pi*i/m for i in range(m)]
	ret = sum( [sin(k*xi)*sin(j*xi) for xi in xs] )
	return 2/m * ret

def right_hand_sin(m, j, k):
	if k!=j:
		return 0
	elif k==0 or j==0:
		return 0
	else:
		return 1
		
ms = [4,6,8]
my_sum = 0
for m in ms:
	for j in range(int(m/2)):
		for k in range(int(m/2)):
			my_sum += abs(left_hand_cos(m,j,k) - 
					right_hand_cos(m,j,k))
			my_sum += abs(left_hand_sin(m,j,k) -
					right_hand_sin(m,j,k))
print(my_sum)
\end{lstlisting}

\end{document}
