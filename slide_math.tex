\documentclass[12pt]{article}

%***************************************************************************************************
% Math
\usepackage{fancyhdr} 
\usepackage{amsfonts}
\usepackage{amsmath}
\usepackage{amssymb}
\usepackage{amsthm}
%\usepackage{dsfont}
\usepackage{color}

%***************************************************************************************************
% Macros
\usepackage{calc}

%***************************************************************************************************
% Commands and Custom Variables	
\newcommand{\problem}[1]{\hspace{-4 ex} \large \textbf{Problem #1} }
\let\oldemptyset\emptyset
\let\emptyset\varnothing
\newcommand{\norm}[1]{\left\lVert#1\right\rVert}
\newcommand{\sint}{\text{s}\kern-5pt\int}
\newcommand{\powerset}{\mathcal{P}}
\renewenvironment{proof}{\hspace{-4 ex} \emph{Proof}:}{\qed}
\newcommand{\RR}{\mathbb{R}}
\newcommand{\NN}{\mathbb{N}}
\newcommand{\QQ}{\mathbb{Q}}
\newcommand{\ZZ}{\mathbb{Z}}
\newcommand{\CC}{\mathbb{C}}

\let\vec\mathbf


%***************************************************************************************************
%page
\usepackage[margin=1in]{geometry}
\usepackage{setspace}
%\doublespacing
\allowdisplaybreaks
\pagestyle{fancy}
\fancyhf{}
\rhead{Shaw \space \thepage}
%\setlength\parindent{0pt}

%***************************************************************************************************
%Code
\usepackage{listings}
\usepackage{courier}
\lstset{
	language=Python,
	showstringspaces=false,
	formfeed=newpage,
	tabsize=4,
	commentstyle=\itshape,
	basicstyle=\ttfamily,
}

%***************************************************************************************************
%Images
\usepackage{graphicx}
\graphicspath{ {images/} }
\usepackage{float}


\title{Project Proposal\\ \large Solving PDEs using Radial Basis Function Finite Difference Methods in Parallel}
\author{Student: Sage Shaw\\ Advisor: Prof. Grady Wright}
\date{April 2, 2018}

%\renewcommand{\abstractname}{Overview}

\usepackage[backend=bibtex, style=numeric, sorting=none]{biblatex}
\addbibresource{pre-proposal.bib}
%\bibliography{SRF}


\begin{document}
	\thispagestyle{empty}
	
	
\section{FD intro}

$$
\nabla^2u_{i,j} \approx \frac{1}{h^2}\left( u_{i-1,j} + u_{i+1, j} + u_{i, j-1} +u_{i, j+1} - 4u_{i,j} \right)
$$

\begin{align*}
	\nabla^2u_{1} & \approx \sum_{n=1}^6 \omega_n u_n \\
	Lu_{1} & \approx \sum_{n=1}^6 \omega_n u_n
\end{align*}

$$
\text{Error} = \mathcal{O}(h^2)
$$

$$
\text{Error} = \mathcal{O}(N^{-1})
$$



\section{Generating weights}
$$
Lu(\vec{y}) \approx \sum_{k=1}^n \omega_k u(\vec{x}_k)
$$

$$
\sum\limits_{k=1}^n \omega_k\phi_k(\vec{x}_i) + \sum\limits_{j=1}^d \lambda_j p_j(\vec{x}_i) = L\phi_i(\vec{y}) 
$$

$$
\{p_k(\vec{x})\}_{k=1}^d = 1, x, y, x^2, xy, y^2, x^3, x^2y, \dots
$$

$$
\sum\limits_{k=1}^n \omega_k p_i(\vec{x}_k) = Lp_i(\vec{x})\vert_{\vec{x}=\vec{y}}
$$

$\text{for } i=1,2,...,n$

$$
\phi_k(\vec{x}) = \phi(\norm{ \vec{x} - \vec{x}_k) }
$$

$\text{for } i=1,2,...,d$

\begin{align*}
	\phi(r) & = r^3 \\
	\phi(r) & = r^4 \ln(r) \\
	\phi(r) & = r^5 \\
	\phi(r) & = r^6 \ln(r) \\
	& \vdots
\end{align*}

	
\section{Solving the System}

$$
\rho(A) \nless 1
$$	

\section{RBF-FD Proof}
$$
\begin{bmatrix}
	\phi_1(\vec{x}_1) & \phi_2(\vec{x}_1) & \dots & \phi_n(\vec{x}_1) & 1 & x_1 & y_1 \\
	\phi_1(\vec{x}_2) & \phi_2(\vec{x}_2) & \dots & \phi_n(\vec{x}_2) & 1 & x_2 & y_2 \\
	\vdots 		& \vdots	& \ddots& \vdots & \vdots& \vdots& \vdots& \\
	\phi_1(\vec{x}_n) & \phi_2(\vec{x}_n) & \dots & \phi_n(\vec{x}_n) & 1 & x_1 & y_1 \\
	1 & 1 & \dots & 1 & 0 & 0 & 0\\
	x_1 & x_2 & \dots & x_n & 0 & 0 & 0\\
	y_1 & y_2 & \dots & y_n & 0 & 0 & 0\\
\end{bmatrix}
\begin{bmatrix}
	\omega_1 \\ \omega_2 \\ \vdots \\ \omega_n \\
	\lambda_1 \\ \lambda_2 \\ \lambda_3
\end{bmatrix}
=
\begin{bmatrix}
	L\phi_1(\vec{y}) \\ L\phi_2(\vec{y}) \\ \vdots \\ L\phi_n(\vec{y}) \\
	1 \\ 0 \\ 0 
\end{bmatrix}
$$

$\text{\phantom{==} for } i=1,2,...,n$

$$
\{\vec{x}_i\}_{i=1}^n
$$
$$
\{u_i\}_{i=1}^n
$$

$$
\sum_{i=1}^n c_i \phi_i(\vec{x}_k) = u(\vec{x}_k) \phantom{=} \forall k
$$
$$
\sum_{=k}^n c_i \phi_i(\vec{y}) = u(\vec{y})
$$

\section{Parallelism}
$$
\begin{bmatrix}
	{\color{red} A_{11} } & {\color{red} A_{12} } & {\color{red} A_{13} } & {\color{red} A_{14} }\\
	{\color{green} A_{21} } & {\color{green} A_{22} } & {\color{green} A_{23} } & {\color{green} A_{24} } \\
	{\color{blue} A_{31} } & {\color{blue} A_{32} } & {\color{blue} A_{33} } & {\color{blue} A_{34} } \\
	{\color{black} A_{41} } & {\color{black} A_{42} } & {\color{black} A_{43} } & {\color{black} A_{44} } \\
\end{bmatrix}
\begin{bmatrix}
	{\color{red} \vec{x}_{1} } \\
	{\color{green} \vec{x}_{2} } \\
	{\color{blue} \vec{x}_{3} } \\
	{\color{black} \vec{x}_{3} } 
\end{bmatrix}
=
\begin{bmatrix}
	{\color{red} \vec{b}_{1} } \\
	{\color{green} \vec{b}_{2} } \\
	{\color{blue} \vec{b}_{3} } \\
	{\color{black} \vec{b}_{3} }
\end{bmatrix}
$$


\end{document}
