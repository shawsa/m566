\documentclass[12pt]{article}

%%%%%%%%%%%%%%%%%%%%%%%%%%%%%%%%%%%%%%%%%%%%%%%%%%%%%%%%%%%%%%%%%%%%%%%%%%%%%%%%%%%%%%%%%%%%%%%%%%%%
% Math
\usepackage{fancyhdr} 
\usepackage{amsfonts}
\usepackage{amsmath}
\usepackage{amssymb}
\usepackage{amsthm}
%\usepackage{dsfont}

%%%%%%%%%%%%%%%%%%%%%%%%%%%%%%%%%%%%%%%%%%%%%%%%%%%%%%%%%%%%%%%%%%%%%%%%%%%%%%%%%%%%%%%%%%%%%%%%%%%%
% Macros
\usepackage{calc}

%%%%%%%%%%%%%%%%%%%%%%%%%%%%%%%%%%%%%%%%%%%%%%%%%%%%%%%%%%%%%%%%%%%%%%%%%%%%%%%%%%%%%%%%%%%%%%%%%%%%
% Commands and Custom Variables	
\newcommand{\problem}[1]{\hspace{-4 ex} \large \textbf{Problem #1} }
\let\oldemptyset\emptyset
\let\emptyset\varnothing
\newcommand{\norm}[1]{\left\lVert#1\right\rVert}
\newcommand{\sint}{\text{s}\kern-5pt\int}
\newcommand{\powerset}{\mathcal{P}}
\renewenvironment{proof}{\hspace{-4 ex} \emph{Proof}:}{\qed}
\newcommand{\RR}{\mathbb{R}}
\newcommand{\NN}{\mathbb{N}}
\newcommand{\QQ}{\mathbb{Q}}
\newcommand{\ZZ}{\mathbb{Z}}
\newcommand{\CC}{\mathbb{C}}
\renewcommand{\Re}{\operatorname{Re}}
\renewcommand{\Im}{\operatorname{Im}}


%%%%%%%%%%%%%%%%%%%%%%%%%%%%%%%%%%%%%%%%%%%%%%%%%%%%%%%%%%%%%%%%%%%%%%%%%%%%%%%%%%%%%%%%%%%%%%%%%%%%
%page
\usepackage[margin=1in]{geometry}
\usepackage{setspace}
%\doublespacing
\allowdisplaybreaks
\pagestyle{fancy}
\fancyhf{}
\rhead{Shaw \space \thepage}
\setlength\parindent{0pt}

%%%%%%%%%%%%%%%%%%%%%%%%%%%%%%%%%%%%%%%%%%%%%%%%%%%%%%%%%%%%%%%%%%%%%%%%%%%%%%%%%%%%%%%%%%%%%%%%%%%%
%Code
\usepackage{listings}
\usepackage{courier}
\lstset{
	language=Python,
	showstringspaces=false,
	formfeed=newpage,
	tabsize=4,
	commentstyle=\itshape,
	basicstyle=\ttfamily,
}

%%%%%%%%%%%%%%%%%%%%%%%%%%%%%%%%%%%%%%%%%%%%%%%%%%%%%%%%%%%%%%%%%%%%%%%%%%%%%%%%%%%%%%%%%%%%%%%%%%%%
%Images
\usepackage{graphicx}
\graphicspath{ {images/} }
\usepackage{float}

%tikz
\usepackage[utf8]{inputenc}
\usepackage{pgfplots}
\usepgfplotslibrary{groupplots}

%%%%%%%%%%%%%%%%%%%%%%%%%%%%%%%%%%%%%%%%%%%%%%%%%%%%%%%%%%%%%%%%%%%%%%%%%%%%%%%%%%%%%%%%%%%%%%%%%%%%
%Hyperlinks
%\usepackage{hyperref}
%\hypersetup{
%	colorlinks=true,
%	linkcolor=blue,
%	filecolor=magenta,      
%	urlcolor=cyan,
%}

\begin{document}
	\thispagestyle{empty}
	
	\begin{flushright}
		Sage Shaw \\
		m566 - Spring 2018 \\
		\today
	\end{flushright}
	
{\large \textbf{HW - Chapter 7}}\bigbreak

\problem{5 (a)} From the text, the eigenvalues of the matrix $A$ are 
$$\lambda_{l,m} = 4 - 2 \big( \cos(l \pi h) + \cos(m \pi h) \big)$$
where $1 \leq l,m \leq N$ and $h = \frac{1}{N+1}$. Note that these are all positive values, and thus $A$ is not just symmetric, but SPD. Then $\norm{A}_2 = \rho(A) = \lambda_{\text{max}}$ and $\norm{A^{-1}}_2 = \rho(A^{-1}) = \lambda_{\text{min}}$. \break

Since the argument to each Cosine function in the formula above is between $0$ and $\pi$ we know that it will be increasing as each $l$ and $m$ increase. Thus the largest eigenvalue will be given by the largest values of $l$ and $m$ and the smallest eigenvalue will be given by the smallest values of $l$ and $m$. Thus 
\begin{align*}
	\lambda_\text{max} & = 4 - 2 \big( \cos(N \pi h) + \cos(N \pi h) \big) \\
	& = 4 - 4 \cos\left(\frac{N }{N+1}\pi \right) \\
	\lambda_\text{min} & = 4 - 2 \big( \cos(1 \pi h) + \cos(1 \pi h) \big) \\
	& = 4 - 4 \cos\left(\frac{1 }{N+1}\pi \right)
\end{align*}
Due to symmetries of Cosine $\cos\left(\frac{N }{N+1}\pi \right) = -\cos\left(\frac{1 }{N+1}\pi \right)$ and we can rewrite
$$
\lambda_\text{max} = 4 + 4 \cos\left(\frac{1 }{N+1}\pi \right)
$$
Finally we find that the condition number of $A$ is given by
\begin{align*}
	\kappa(A) & = \left( 4 + 4 \cos\left(\frac{1 }{N+1}\pi \right) \right) 
		\left( 4 - 4 \cos\left(\frac{1 }{N+1}\pi \right) \right)^{-1} \\
	& = \frac{1 + \cos\left(\frac{1 }{N+1}\pi \right)}{ 1 - \cos\left(\frac{1 }{N+1}\pi \right)}
\end{align*}
As $N$ gets large, the numerator approaches $2$ and the denominator approaches $0$, thus the condition number gets large. We can verify this by using Taylor Series approximations
\begin{align*}
	\frac{1 + \cos(x)}{ 1 - \cos(x)} & \approx \frac{1 + 1 - \frac{x^2}{2}}{1 - 1 + \frac{x^2}{2}} \\
	& = \frac{2 - \frac{x^2}{2}}{\frac{x^2}{2}} \\
	& = \frac{4}{x^2} - 1 \\
	\kappa(A) & \approx \left( \frac{2}{\frac{1}{N+1}\pi} \right)^2 \\
	& = \left( \frac{2N + 2}{\pi} \right)^2 \\
	& = \mathcal{O}(N^2) \\
	& = \mathcal{O}(n)
\end{align*}
As expected the condition number scales linearly with the discretization.


\end{document}
