\documentclass[12pt]{article}

%%%%%%%%%%%%%%%%%%%%%%%%%%%%%%%%%%%%%%%%%%%%%%%%%%%%%%%%%%%%%%%%%%%%%%%%%%%%%%%%%%%%%%%%%%%%%%%%%%%%
% Math
\usepackage{fancyhdr} 
\usepackage{amsfonts}
\usepackage{amsmath}
\usepackage{amssymb}
\usepackage{amsthm}
\usepackage{enumitem}
%\usepackage{dsfont}

%%%%%%%%%%%%%%%%%%%%%%%%%%%%%%%%%%%%%%%%%%%%%%%%%%%%%%%%%%%%%%%%%%%%%%%%%%%%%%%%%%%%%%%%%%%%%%%%%%%%
% Macros
\usepackage{calc}

%%%%%%%%%%%%%%%%%%%%%%%%%%%%%%%%%%%%%%%%%%%%%%%%%%%%%%%%%%%%%%%%%%%%%%%%%%%%%%%%%%%%%%%%%%%%%%%%%%%%
% Commands and Custom Variables	
\newcommand{\problem}[1]{\hspace{-4 ex} \large \textbf{Problem #1} }
\let\oldemptyset\emptyset
\let\emptyset\varnothing
\newcommand{\norm}[1]{\left\lVert#1\right\rVert}
\newcommand{\sint}{\text{s}\kern-5pt\int}
\newcommand{\powerset}{\mathcal{P}}
\renewenvironment{proof}{\hspace{-4 ex} \emph{Proof}:}{\qed}
\newcommand{\RR}{\mathbb{R}}
\newcommand{\NN}{\mathbb{N}}
\newcommand{\QQ}{\mathbb{Q}}
\newcommand{\ZZ}{\mathbb{Z}}
\newcommand{\CC}{\mathbb{C}}
\renewcommand{\Re}{\operatorname{Re}}
\renewcommand{\Im}{\operatorname{Im}}

\newcommand{\solution}{\vspace{2 ex} \hspace{-5 ex} \emph{Solution.} }


%%%%%%%%%%%%%%%%%%%%%%%%%%%%%%%%%%%%%%%%%%%%%%%%%%%%%%%%%%%%%%%%%%%%%%%%%%%%%%%%%%%%%%%%%%%%%%%%%%%%
%page
\usepackage[margin=1in]{geometry}
\usepackage{setspace}
%\doublespacing
\allowdisplaybreaks
\pagestyle{fancy}
\fancyhf{}
\rhead{Malmuth \& Shaw \space \thepage}
\setlength\parindent{0pt}

%%%%%%%%%%%%%%%%%%%%%%%%%%%%%%%%%%%%%%%%%%%%%%%%%%%%%%%%%%%%%%%%%%%%%%%%%%%%%%%%%%%%%%%%%%%%%%%%%%%%
%Code
\usepackage{listings}
\usepackage{courier}
\lstset{
	language=Python,
	showstringspaces=false,
	formfeed=newpage,
	tabsize=4,
	commentstyle=\itshape,
	basicstyle=\ttfamily,
}

%%%%%%%%%%%%%%%%%%%%%%%%%%%%%%%%%%%%%%%%%%%%%%%%%%%%%%%%%%%%%%%%%%%%%%%%%%%%%%%%%%%%%%%%%%%%%%%%%%%%
%Images
\usepackage{graphicx}
\graphicspath{ {images/} }
\usepackage{float}

%tikz
\usepackage[utf8]{inputenc}
\usepackage{pgfplots}
\usepgfplotslibrary{groupplots}

%%%%%%%%%%%%%%%%%%%%%%%%%%%%%%%%%%%%%%%%%%%%%%%%%%%%%%%%%%%%%%%%%%%%%%%%%%%%%%%%%%%%%%%%%%%%%%%%%%%%
%Hyperlinks
%\usepackage{hyperref}
%\hypersetup{
%	colorlinks=true,
%	linkcolor=blue,
%	filecolor=magenta,      
%	urlcolor=cyan,
%}

\begin{document}
	\thispagestyle{empty}
	
	\begin{flushright}
		Daniel Malmuth \& Sage Shaw \\
		m566 - Spring 2018 \\
		\today
	\end{flushright}
	
{\large \textbf{HW - Chapter 7}}\bigbreak

\problem{5 (a)} From the text, the eigenvalues of the matrix $A$ are 
$$\lambda_{l,m} = 4 - 2 \big( \cos(l \pi h) + \cos(m \pi h) \big)$$
where $1 \leq l,m \leq N$ and $h = \frac{1}{N+1}$. Note that these are all positive values, and thus $A$ is not just symmetric, but SPD. Then $\norm{A}_2 = \rho(A) = \lambda_{\text{max}}$ and $\norm{A^{-1}}_2 = \rho(A^{-1}) = \lambda_{\text{min}}$. \break

Since the argument to each Cosine function in the formula above is between $0$ and $\pi$ we know that it will be increasing as each $l$ and $m$ increase. Thus the largest eigenvalue will be given by the largest values of $l$ and $m$ and the smallest eigenvalue will be given by the smallest values of $l$ and $m$. Thus 
\begin{align*}
	\lambda_\text{max} & = 4 - 2 \big( \cos(N \pi h) + \cos(N \pi h) \big) \\
	& = 4 - 4 \cos\left(\frac{N }{N+1}\pi \right) \\
	\lambda_\text{min} & = 4 - 2 \big( \cos(1 \pi h) + \cos(1 \pi h) \big) \\
	& = 4 - 4 \cos\left(\frac{1 }{N+1}\pi \right)
\end{align*}
Due to symmetries of Cosine $\cos\left(\frac{N }{N+1}\pi \right) = -\cos\left(\frac{1 }{N+1}\pi \right)$ and we can rewrite
$$
\lambda_\text{max} = 4 + 4 \cos\left(\frac{1 }{N+1}\pi \right)
$$
Finally we find that the condition number of $A$ is given by
\begin{align*}
	\kappa(A) & = \left( 4 + 4 \cos\left(\frac{1 }{N+1}\pi \right) \right) 
		\left( 4 - 4 \cos\left(\frac{1 }{N+1}\pi \right) \right)^{-1} \\
	& = \frac{1 + \cos\left(\frac{1 }{N+1}\pi \right)}{ 1 - \cos\left(\frac{1 }{N+1}\pi \right)} \\
	& = \cot^2\left(\frac{\pi }{2(N+1)} \right)
\end{align*}


As $N$ gets large, the numerator approaches $2$ and the denominator approaches $0$, thus the condition number gets large. We can verify this by using Taylor Series approximations
\begin{align*}
	\frac{1 + \cos(x)}{ 1 - \cos(x)} & \approx \frac{1 + 1 - \frac{x^2}{2}}{1 - 1 + \frac{x^2}{2}} \\
	& = \frac{2 - \frac{x^2}{2}}{\frac{x^2}{2}} \\
	& = \frac{4}{x^2} - 1 \\
	\kappa(A) & \approx \left( \frac{2}{\frac{1}{N+1}\pi} \right)^2 \\
	& = \left( \frac{2N + 2}{\pi} \right)^2 \\
	& = \mathcal{O}(N^2) \\
	& = \mathcal{O}(n)
\end{align*}
As expected the condition number scales linearly with the discretization.

\bigbreak
%%%%%%%%%%%%%%%%%%%%%%%%%%%%%%%%%%%%%%%%%%%%%%%%%%%%%%%%%%%%%%%%%%%%%%%%%%%%%%%%%%%%%%%%%%%%%%%%%%%%

\problem{5 (b)} Since this problem asks to write code, it will be included here instead of in an appendix.
\begin{lstlisting}
def gen_A(N):
	n = N**2
	A = np.zeros((n,n))
	A += np.diag(4*np.ones(n), k=0)
	A += np.diag(-1*np.ones(n-1), k=1)
	A += np.diag(-1*np.ones(n-1), k=-1)
	A += np.diag(-1*np.ones(n-N), k=N)
	A += np.diag(-1*np.ones(n-N), k=-N)
	for i in range(N,n,N):
		A[i,i-1] = 0
		A[i-1,i] = 0
	return A

def residual(x,b):
	n = len(x)
	N = int(np.sqrt(n))
	assert N**2 == n
	assert N>= 3
	r = np.zeros(n)
	r[0] = b[0] -4*x[0] + x[1] + x[N]
	for i in range(1, N):
		r[i] = b[i] + x[i-1] - 4*x[i] + x[i+1] + x[i+N]
	for i in range(N, n-N):
		r[i] = b[i] + x[i-N] + x[i-1] - 4*x[i] + x[i+1] 
+ x[i+N]
	for i in range(n-N, n-1):
		r[i] = b[i] + x[i-N] + x[i-1] - 4*x[i] + x[i+1]
	r[-1] = b[-1] + x[-1-N] + x[-2] - 4*x[-1]
	for i in range(N,n,N):
		r[i] -= x[i-1]
	for i in range(N-1,n-1,N):
		r[i] -= x[i+1]
	return r

def jacobi_solve(x, b, tol, max_iter=10**9):
	n = len(x)
	N = int(np.sqrt(n))
	assert N**2 == n
	assert N>= 3
	x_old = x
	r = residual(x_old, b)
	res_norms = [np.linalg.norm(r)]
	x_new = x_old + r/4
	iterations = 1
	while np.linalg.norm(r) > np.linalg.norm(b)*tol and
 iterations < max_iter:
		r = residual(x_new, b)
		res_norms.append(np.linalg.norm(r))
		x_new, x_old = x_new + r/4, x_new
		iterations += 1
	cond = (1+np.cos(np.pi/(N+1)))/(1-np.cos(np.pi/(N+1)))
	return x_new, iterations, cond, res_norms

def p5b(N, tol):
	N=3
	tol = 10**-5
	n = N**2
	x_old = np.ones(n)
	b = np.ones(n)/(N+1)
	r = residual(x_old, b)
	x_new = x_old + r/4
	while np.linalg.norm(r) > np.linalg.norm(b)*tol:
		r = residual(x_new, b)
		x_new, x_old = x_new + r/4, x_new
	print(x_new)
	#test the result by checking the magnitude of the residual
	print(np.linalg.norm(b - np.dot(gen_A(N), x_new)))
	cond = (1+np.cos(np.pi/(N+1)))/(1-np.cos(np.pi/(N+1)))
	print('The condition number is %g.' % cond)
\end{lstlisting}

The output for this code ran with $N = 3$ and $\text{tol} = 10^{-8}$ is as expected.
\begin{lstlisting}
	>>> p5b()
	[0.17187613 0.21875149 0.17187613 0.21875149 0.28125226 0.21875149
	0.17187613 0.21875149 0.17187613]
	5.234360313439405e-06
	The condition number is 5.82843.
\end{lstlisting}

\bigbreak
%%%%%%%%%%%%%%%%%%%%%%%%%%%%%%%%%%%%%%%%%%%%%%%%%%%%%%%%%%%%%%%%%%%%%%%%%%%%%%%%%%%%%%%%%%%%%%%%%%%%

\problem{5 (c)} The results for using the Jacobi method for $N = 2^l - 1$ are given below.
	\begin{center}
		\begin{tabular}{|c|c|c|c|c|c|}
			\hline
			$l$&$N$&$n$&iterations required&error at $(0.5,0.5)$&$\kappa(A)$\\ \hline
			2&3&9&43&3.390e-07&5.828e+00\\ \hline
			3&7&49&182&7.631e-07&2.527e+01\\ \hline
			4&15&225&738&9.067e-07&1.031e+02\\ \hline
			5&31&961&2956&9.722e-07&4.143e+02\\ \hline
		\end{tabular}
	\end{center}



\begin{figure}[H]
	\caption{$n$ versus $\kappa(A)$}
	\includegraphics[width=1\textwidth]{hwch7_figure_1}
	\label{5c1}
	\centering
\end{figure}

\begin{figure}[H]
	\caption{Jacobi Method Performance $N=31, n=31^2$}
	\includegraphics[width=1\textwidth]{hwch7_figure_2}
	\label{5c2}
	\centering
\end{figure}

As expected the condition number scales linearly with $n$ and it takes roughly a constant number of iterations to reduce the norm of the residual by a factor. From the table it seems that the number of iterations required scales linearly with $n$. 

\bigbreak
%%%%%%%%%%%%%%%%%%%%%%%%%%%%%%%%%%%%%%%%%%%%%%%%%%%%%%%%%%%%%%%%%%%%%%%%%%%%%%%%%%%%%%%%%%%%%%%%%%%%

\problem{9 (a)} Using the iteration scheme $x_{k+1} = x_k + \alpha(b-Ax_k)$ the splitting associated would be $A= M-N$ where $M^{-1} = \alpha I$. Then $M = \frac{1}{\alpha}I$ and
\begin{align*}
	T & = I - M^{-1}A \\
	& = I - \alpha I A \\
	& = I - \alpha A
\end{align*}

\bigbreak
%%%%%%%%%%%%%%%%%%%%%%%%%%%%%%%%%%%%%%%%%%%%%%%%%%%%%%%%%%%%%%%%%%%%%%%%%%%%%%%%%%%%%%%%%%%%%%%%%%%%

\problem{9 (b)} We have a theorem that says that a Stationary Method converges if and only if the spectral radius is less than one. Suppose that $A$ is SPD with eigenvalues $\lambda_1 > \lambda_2 > ... > \lambda_n > 0$. Then $T$ is symmetric. Suppose that $x_k$ is an eigenvector corresponding to the eigenvalue $\lambda_k$. Then
\begin{align*}
	Tx_k & = Ix_k - \alpha Ax_k \\
	& = x_k - \lambda_k \alpha x_k \\
	& = \left(1- \lambda_k \alpha \right)x_k
\end{align*}
Thus each eigenvector of $A$ is also an eigenvector of $T$ corresponding to an eigenvalue of $\left( 1- \lambda_k \alpha \right)$. Since $T$ is symmetric, $\rho(T) = \max\limits_k \{ \left\vert 1 - \lambda_k \alpha \right \vert \}$. In order for the method to converge, we require that this value be less than 1. \bigbreak

Suppose that $\alpha < 0$. Then we get that 
$$
- \lambda_1 \alpha > - \lambda_2 \alpha > ... > - \lambda_n \alpha > 0
$$
$$
1 - \lambda_1 \alpha > 1 - \lambda_2 \alpha > ... > 1 - \lambda_n \alpha > 1
$$
and our iterations fail to converge.
Suppose that $\alpha > 0$. Then
$$
1 - \lambda_1 \alpha < 1 - \lambda_2 \alpha < ... > 1 - \lambda_n \alpha < 1
$$
As long as $-1 < 1 - \lambda_1 \alpha $ this will converge. Solving for $\alpha$ we get our final result: \bigbreak

The iteration $x_{k+1} = x_k + \alpha(b-Ax_k)$ will converge if and only if $\frac{2}{\lambda_1} < \alpha$ where $\lambda_1$ is the largest eigenvalue of the SPD matrix $A$. \bigbreak

To maximize the rate of convergence we need to minimize the spectral radius. This is made simpler by noting that 
$$
\rho(T) = \max \left\{  \left\vert 1- \lambda_1\alpha \right\vert, \left\vert 1 - \lambda_2\alpha\right\vert \right \}
$$
since the eigenvalues are all between these two. Since $1-\lambda_1\alpha < 1 - \lambda_n\alpha$ we really only need to consider
$$
\rho(T) = \max \left\{  \lambda_1\alpha - 1 , 1 - \lambda_n\alpha \right \}
$$
In addition, since $\lambda_1\alpha - 1$ is increasing with $\alpha$ and $1 - \lambda_n\alpha$ is decreasing with $\alpha$, the $\rho(T)$ will reach a minimum at the intersection point. That is that the value of $\alpha$ that minimizes $\rho(T)$ satisfies
\begin{align*}
	\lambda_1\alpha - 1 & = 1 - \lambda_n\alpha \\
	(\lambda_1 + \lambda_n )\alpha & = 2 \\
	\alpha & = \frac{2}{\lambda_1 + \lambda_2}
\end{align*}
We can find the spectral radius in this case by
\begin{align*}
	\rho(T) & = lambda_1\alpha - 1 \\
	\rho(T) & = 1 - \lambda_n\alpha \\
	2 \rho(T) & = \alpha(\lambda_1 - \lambda_n) \\
	2 \rho(T) & = 2 \frac{\lambda_1 - \lambda_n}{\lambda_1 + \lambda_n} \\
	\rho(T) & = \frac{\kappa(A) - 1}{\kappa(A)+1}
\end{align*}
As $\kappa(A)$ gets large $\rho(T)$ approaches $1$ and converges slowly. 

\bigbreak
%%%%%%%%%%%%%%%%%%%%%%%%%%%%%%%%%%%%%%%%%%%%%%%%%%%%%%%%%%%%%%%%%%%%%%%%%%%%%%%%%%%%%%%%%%%%%%%%%%%%

\problem{9 (c)} The statement in the text is false. If $A$ is strictly diagonally dominant and $\alpha = 1$ the iterative scheme will only converge if $\lambda_1 > 2$

\bigbreak
%%%%%%%%%%%%%%%%%%%%%%%%%%%%%%%%%%%%%%%%%%%%%%%%%%%%%%%%%%%%%%%%%%%%%%%%%%%%%%%%%%%%%%%%%%%%%%%%%%%%

\problem{13 } The plot of the condition number of $A$ with $n$ is exactly the same as in problem 5 since the condition number does not depend on any method of solving the system. 

Since it was explicitly asked for in this problem, the code used to generate the results appears here instead of in an Appendix.

\begin{lstlisting}
def A_mult(x):
	n = len(x)
	N = int(np.sqrt(n))
	assert N**2 == n
	assert N>= 3
	r = np.zeros(n)
	r[0] = 4*x[0] - x[1] - x[N]
	for i in range(1, N):
		r[i] = -x[i-1] + 4*x[i] - x[i+1] - x[i+N]
	for i in range(N, n-N):
		r[i] = -x[i-N] - x[i-1] + 4*x[i] - x[i+1] - x[i+N]
	for i in range(n-N, n-1):
		r[i] = -x[i-N] - x[i-1] + 4*x[i] - x[i+1]
	r[-1] = -x[-1-N] - x[-2] + 4*x[-1]
	for i in range(N,n,N):
		r[i] += x[i-1]
	for i in range(N-1,n-1,N):
		r[i] += x[i+1]
	return r

def cg_solve(x, b, tol, max_iter=10**9):
	n = len(x)
	N = int(np.sqrt(n))
	assert N**2 == n
	assert N>= 3
	x_old = x
	r = residual(x_old, b)
	res_norms = [np.linalg.norm(r)]
	iterations = 1
	delta = np.dot(r, r)
	b_delta = np.dot(b,b)
	p = r
	while delta > b_delta * tol**2 and iterations < max_iter:
		s = A_mult(p)
		alpha = delta/np.dot(p,s)
		x_new = x_old + alpha*p
		r -= alpha * s
		res_norms.append(np.linalg.norm(r))
		delta_new = np.dot(r,r)
		p = r + delta_new/delta * p
		x_old, delta = x_new, delta_new
		iterations += 1
	cond = (1+np.cos(np.pi/(N+1)))/(1-np.cos(np.pi/(N+1)))
	return x_new, iterations, cond, res_norms
\end{lstlisting}

	\begin{center}
		\begin{tabular}{|c|c|c|c|c|c|}
			\hline
			$l$&$N$&$n$&iterations required&error at $(0.5,0.5)$&$\kappa(A)$\\ \hline
			2&3&9&21&3.339e-07&5.828e+00\\ \hline
			3&7&49&31&2.267e-07&2.527e+01\\ \hline
			4&15&225&54&3.442e-08&1.031e+02\\ \hline
			5&31&961&91&1.761e-07&4.143e+02\\ \hline
			6&63&3969&165&-2.676e-08&1.659e+03\\ \hline
		\end{tabular}
	\end{center}
		


\begin{figure}[H]
	\caption{Conjugate Gradient Performance $N=31, n=31^2$}
	\includegraphics[width=1\textwidth]{hwch7_figure_3_cg_N31_2}
	\label{13c}
	\centering
\end{figure}

As $n$ gets larger the number of iterations drops dramatically and is always less than $n$. This is far better than the Jacobi Method. 

\bigbreak
%%%%%%%%%%%%%%%%%%%%%%%%%%%%%%%%%%%%%%%%%%%%%%%%%%%%%%%%%%%%%%%%%%%%%%%%%%%%%%%%%%%%%%%%%%%%%%%%%%%%

\problem{25 (a)} Applying finite differences to the Helmoholtz equation we get
\begin{align*}
	4u_{i,j} - u_{i+1,j} - u_{i-1,j} - u)_{i,j+1} - u_{i,j-1} - \omega^2 h^2 u_{i,j} & = b_{i,j} \\
	(4- (\omega h)^2)u_{i,j} - u_{i+1,j} - u_{i-1,j} - u_{i,j+1} - u_{i,j-1} & = b_{i,j}
\end{align*}
where $b_{i,j} = h^2g(ih,jh)$ as in the text. This leads to an $A$ matrix similar to example 7.1 except that it has $4-(\omega h)^2$ for each diagonal element entry instead of $4$ for each diagonal entry. \bigbreak

\bigbreak
%%%%%%%%%%%%%%%%%%%%%%%%%%%%%%%%%%%%%%%%%%%%%%%%%%%%%%%%%%%%%%%%%%%%%%%%%%%%%%%%%%%%%%%%%%%%%%%%%%%%

\problem{25 (b)} If we let $A^\prime$ represent the matrix from example 7.1 (the one with $4$ along the diagonal) then our matrix would be given by $A = A^\prime- I(\omega h)^2$. Suppose that $\lambda$ is an eigenvalue of $A$. Then
\begin{align*}
	0 & = A - \lambda I \\
	& = A - (\lambda + (\omega h)^2 - (\omega h)^2)I \\
	& = A - \lambda I - (\omega h)^2I + (\omega h)^2I \\
	& = (A + (\omega h)^2I) - (\lambda + (\omega h)^2)I \\
	& = A^\prime - (\lambda + (\omega h)^2)I
\end{align*}
Thus concluding that $\lambda + (\omega h)^2$ is an eigenvalue of $A^\prime$. This tells us that all of the eigenvalues of $A$ are given by
$$
\lambda_{l,m} = 4 - 2(\cos(l \pi h) + \cos(m \pi h)) - (\omega h)^2, 1\leq l,m \leq N
$$

In order for $A$ to be positive definite, we need need for all of the eigenvalues to remain positive. Thus the $\omega_c^2h^2$ we seek is simply the smallest eigenvalue of $A^\prime$. Thus
\begin{align*}
	\omega_c^2h^2 & = 4 - 4 \cos\left(\frac{1 }{N+1}\pi \right) \\
	\omega_c & = \frac{2}{h} \sqrt{1- \cos\left(\frac{1 }{N+1}\pi \right)}
\end{align*}

\bigbreak
%%%%%%%%%%%%%%%%%%%%%%%%%%%%%%%%%%%%%%%%%%%%%%%%%%%%%%%%%%%%%%%%%%%%%%%%%%%%%%%%%%%%%%%%%%%%%%%%%%%%

\problem{25 (c)} Code for this part was taken largely from the author's code for example 7.7, though some modifications were made. This was tested using the dampened Jacobi method for both relaxation steps, and again using MinRes for both relaxation steps. The results were identical. Both methods were effective at reducing oscillations in the residual allowing the multigrid method to quickly reduce the smoothed residual. The number of multigrid cycles required did not change significantly with $N$.

\begin{figure}[H]
	\caption{Multigrid method applied to the Helmoholtz equation with $\omega = 1$}
	\includegraphics[width=.75\textwidth]{hwch7_figure_4_multigrid_w1_jacobi}
	\label{multigrid}
	\centering
\end{figure}

On checking the norms of the residuals we can see that they are way below the set tolerance further emphasizing the impressive convergence speed of multigrid methods. In the table below, $l$ is given along with the norm of the residual where $N = 2^l-1$.

	\begin{center}
		\begin{tabular}{|c|c|c|c|}
			\hline
			$l$ & $N$ & $n$ & $\norm{r}$\\ \hline
			4& 15 & 225 & 9.086750e-09\\ \hline
			5& 31 & 961 & 2.824305e-08\\ \hline
			6& 63 & 3969 & 4.440823e-09\\ \hline
			7& 127 & 16129 & 3.003766e-09\\ \hline
		\end{tabular}
	\end{center}
	
In the case where $\omega = 10$, we can see by part (b) of this problem that $\omega_c = 4.44277$ which is smaller than the prescribed $\omega=10$ for the second part of this problem. This means that $A$ is no longer positive definite and thus the multigrid method will not work. This is experimentally verified by changing $\omega$ to 10 in the code given in the appendix. In light of this, we turn to MinRes which does not require that the matrix be positive definite. 
\bigbreak

We find that MinRes requires $199$ iterations to converge for this particular problem. In contrast, the multigrid method required 4 iterations per recursion making a total of 16 relaxation steps plus solving the $8 \times 8$ system, and the interpolation steps. All of this together is far faster than the $199$ iterations of MinRes. 


\bigbreak
%%%%%%%%%%%%%%%%%%%%%%%%%%%%%%%%%%%%%%%%%%%%%%%%%%%%%%%%%%%%%%%%%%%%%%%%%%%%%%%%%%%%%%%%%%%%%%%%%%%%

\problem{16} Let $ A $ be symmetric positive definite and consider the CG method. Show that for $ \textbf{r}_k $ the residual in the $ k $th iteration and $ \textbf{e}_k $ the error in the $ k $th iteration, the following energy norm identities hold:
\begin{enumerate}[leftmargin=0.6cm,label=(\alph*)]
	\item $ \lVert \textbf{r}_k \rVert_{A^{-1}} = \lVert \textbf{e}_k \rVert_{A}$.
	\item If $ \textbf{x}_k $ minimizes the quadratic function $ \phi(\textbf{x}) = \frac{1}{2}\textbf{x}^TA\textbf{x} - \textbf{x}^T\textbf{b} $ (note that \textbf{x} here is an argument vector, not the exact solution) over a subspace $ S $, then the same $ \textbf{x}_k $ minimizes the error $ \lVert \textbf{e}_k \rVert_{A} $ over $ S $.
\end{enumerate}

\vspace{-2 ex} \solution
\begin{enumerate}[leftmargin=0.6cm,label=(\alph*)]
	\item Since $ A $ is symmetric, we have that $ A^{-1} $ is symmetric by uniqueness of inverses. Additionally, since $ A $ is positive definite, $ A $ only have positive eigenvalues. Since the eigenvalues for $ A^{-1} $ are found using the reciprocals of the eigenvalues of $ A $, all eigenvalues for $ A^{-1} $ are positive and thus $ A^{-1} $ is positive definite. We have that $ A^{-1} $ is symmetric positive definite, thus the associated energy norm of $ \textbf{r}_k $ is
	\begin{align*}
	\lVert \textbf{r}_k \rVert_{A^{-1}} = \sqrt{\textbf{r}_k^TA^{-1}\textbf{r}_k}.
	\end{align*}
	We establish the relationship between $ \textbf{r}_k $ and $ \textbf{e}_k $ by
	\begin{align*}
	\textbf{r}_k &= \textbf{b} - A\textbf{x}_k \\
	&= A\textbf{x} - A\textbf{x}_k \\
	&= A(\textbf{x}-\textbf{x}_k) \\
	&= A\textbf{e}_k.
	\end{align*}
	This also implies that $ \textbf{e}_k = A^{-1}\textbf{r}_k $. Therefore,
	\begin{align*}
	\lVert \textbf{r}_k \rVert_{A^{-1}} &= \sqrt{\textbf{r}_k^TA^{-1}\textbf{r}_k} \\
	&= \sqrt{\textbf{r}_k^T\textbf{e}_k} \\
	&= \sqrt{(A\textbf{e}_k)^T\textbf{e}_k} \\
	&= \sqrt{\textbf{e}_k^TA^T\textbf{e}_k} \\
	&= \sqrt{\textbf{e}_k^TA\textbf{e}_k} \\
	&= \lVert \textbf{e}_k \rVert_{A}.  
	\end{align*} 
	\item Since $ \textbf{x}_k $ minimizes the quadratic function $ \phi $, we have that 
	\begin{align*}
	\nabla \phi(\textbf{x}_k) = A\textbf{x}_k - \textbf{b} = -\textbf{r}_k = 0, 
	\end{align*}
	i.e. $ \textbf{r}_k = \textbf{0} $. This means that
	\begin{align*}
	\lVert \textbf{r}_k \rVert_{A^{-1}} = \sqrt{\textbf{r}_k^TA^{-1}\textbf{r}_k} = 0.
	\end{align*}
	By (a), we know that $ \lVert \textbf{r}_k \rVert_{A^{-1}} = \lVert \textbf{e}_k \rVert_{A}$, thus $ \lVert \textbf{e}_k \rVert_{A} = 0 $. Therefore, $ \textbf{x}_k $ minimizes $ \lVert \textbf{e}_k \rVert_{A} $. 
\end{enumerate}

\bigbreak
%%%%%%%%%%%%%%%%%%%%%%%%%%%%%%%%%%%%%%%%%%%%%%%%%%%%%%%%%%%%%%%%%%%%%%%%%%%%%%%%%%%%%%%%%%%%%%%%%%%%

\problem{23} Define a linear problem with $ n = 500 $ using the script
\begin{lstlisting}[language=MATLAB]
A = randn(500,500); xt = randn(500,1); b = A*xt;
\end{lstlisting}
Now we save \textbf{xt} away and solve $ A\textbf{x} = \textbf{b} $. Set $ \text{tol} = 1.e-6 $ and maximum iteration limit of 2000. Run three solvers for this problem:
\begin{enumerate}[leftmargin=0.6cm,label=(\alph*)]
	\item CG on the normal equations: $ A^TA\textbf{x} = A^T\textbf{b} $.
	\item GMRES(500)
	\item GMRES(100), i.e., restarted GMRES with $ m = 100 $. 
\end{enumerate}
Record residual norm and solution error norm for each run.

What are your conclusions?

\solution
\begin{enumerate}[leftmargin=0.6cm,label=(\alph*)]
	\item Residual norm: 0.0095; Solution error norm: 0.0758 (runs 683 iterations and converges)
	\item Residual norm: 0; Solution error norm: 8.1894e-12 (runs 862 iterations and converges)
	\item Residual norm: 0.82; Solution error norm: 23.3632 (runs 2000 iterations and does not converge)
\end{enumerate}
We find that the CG method converges the quickest and that the GMRES converges at a similar rate. Additionally, by using the GMRES, we find the smallest residual norm as well as the smallest solution error norm. As stated in the book, the GMRES is a very strong tool to use when applied to matrices that don't require much storage space. Although our repeated GMRES did not converge within 2000 iterations, it would likely see more practical use with larger, sparse matrices. 


\bigbreak

\problem{19}



\bigbreak
%%%%%%%%%%%%%%%%%%%%%%%%%%%%%%%%%%%%%%%%%%%%%%%%%%%%%%%%%%%%%%%%%%%%%%%%%%%%%%%%%%%%%%%%%%%%%%%%%%%%

{\hspace{-4 ex} \huge \textbf{Appendix - Code listings}}\bigbreak

\large{Problem 25(c) Multigrid code:}
\begin{lstlisting}[language=MATLAB]
clear all
close all
% This code is a modified version of the code used in example 7.7
% https://www.mathworks.com/support/books/book69732.html

l = 7;
ns = zeros(4,1);
iterations = zeros(4,1);
for i=1:4
l = i+3;
N = 2^l - 1;

%N = 2^7 -1;
tol = 1.e-6;  % should actually depend on N but never mind.


w = 1;
h = 1/(N+1);
n = N^2;
ns(i,1) = n;
A = delsq(numgrid('S',N+2)) - diag(ones(n,1).*(w*h)^2);
%n = size(A,1);
b = ones(n,1)*h^2;

% Solve using multigrid
xmg = zeros(n,1); bb = norm(b);
flevel = log2(N+1);
for itermg = 1:30
	[xmg,res] = poismg(A,b,xmg,flevel);
	if res/bb < tol
		break;
	end
end
%[xmg, ~, ~, itermg] = minres(A,b,tol,1000);
iterations(i,1) = itermg;
fprintf("%i: %e\n", l, norm(A*xmg - b));
end

hold on
plot(ns, iterations,'o');
xlabel("n");
ylabel("Mutigrid Iterations");
ylim([0,10]);
hold off
\end{lstlisting}

\large{Problem 25 (c) MinRes Code}
\begin{lstlisting}[language=MATLAB]
N = 2^7-1;
w = 1;
h = 1/(N+1);
n = N^2;
A = delsq(numgrid('S',N+2)) - diag(ones(n,1).*(w*h)^2);
b = ones(n,1)*h^2;

[x, ~, ~, iter] = minres(A,b,1e-6,2000);
norm(A*x-b)
iter
\end{lstlisting}

\large{Code for the poismg function}
\begin{lstlisting}[language=MATLAB]
% This code is a modified version of the code used in example 7.7
% https://www.mathworks.com/support/books/book69732.html


function [x,res] = poismg(A,b,x,level,tol)
%
% function [x,res] = poismg(A,b,x,level)
%
% multigrid V-cycle to solve simplest Poisson on a square
% The uniform grid is N by N, N = 2^l-1 some l > 2,
% b is the right hand side; homogeneous Dirichlet;
% A has been created by   A = delsq(numgrid('S',N+2));

coarsest = 3;              % coarsest grid
nu1 = 2;                   % relaxations before coarsening grid
nu2 = 2;                   % relaxations after return to finer level
omeg = .8;                 % relaxation damping parameter

if level == coarsest
x = A\b;               % solve exactly on coarsest level
r = b - A*x;

else % begin multigrid cycle

% relax using damped Jacobi
Dv = diag(A);         % diagonal part of A as vector
for i=1:nu1
	r = b - A*x;
	x = x + omeg*r./Dv;
end
% relax using minres
%x = minres(A,b,10,tol,nu1);

% restrict residual from r to rc on coarser grid
r = b - A*x; 
N = sqrt(length(b));
r = reshape(r,N,N);
Nc = (N+1)/2 - 1; nc = Nc^2;    % coarser grid dimensions
Ac = delsq(numgrid('S',Nc+2));  % coarser grid operator
rc = r(2:2:N-1,2:2:N-1) + .5*(r(3:2:N,2:2:N-1)+r(1:2:N-2,2:2:N-1) +...
	r(2:2:N-1,3:2:N)+r(2:2:N-1,1:2:N-2)) + .25*(r(3:2:N,3:2:N)+...
	r(3:2:N,1:2:N-2)+r(1:2:N-2,3:2:N)+r(1:2:N-2,1:2:N-2));
rc = reshape(rc,nc,1);

% descend level. Use V-cycle
vc = zeros(size(rc));            % initialize correction to 0
[vc,r] = poismg(Ac,rc,vc,level-1); % samesame on coarser grid

% prolongate correction from vc to v on finer grid
v = zeros(N,N);
vc = reshape(vc,Nc,Nc);
v(2:2:N-1,2:2:N-1) = vc;
vz = [zeros(1,N);v;zeros(1,N)];   % embed v with a ring of 0s
vz = [zeros(N+2,1),vz,zeros(N+2,1)];
v(1:2:N,2:2:N-1) = .5*(vz(1:2:N,3:2:N)+vz(3:2:N+2,3:2:N));
v(2:2:N-1,1:2:N) = .5*(vz(3:2:N,1:2:N)+vz(3:2:N,3:2:N+2));
%v(3:2:N-2,3:2:N-2) = .25*(v(2:2:N-3,2:2:N-3)+v(2:2:N-3,4:2:N-1)+...
%    v(4:2:N-1,4:2:N-1)+v(4:2:N-1,2:2:N-3));
v(1:2:N,1:2:N) = .25*(vz(1:2:N,1:2:N)+vz(1:2:N,3:2:N+2)+...
	vz(3:2:N+2,3:2:N+2)+vz(3:2:N+2,1:2:N));

% add to current solution
n = N^2;
x = x + reshape(v,n,1);

% relax using damped Jacobi
for i=1:nu2
	r = b - A*x;
	x = x + omeg*r./Dv;
end
% relax using minres
%x = minres(A,b,10,tol,nu2);

end
res = norm(b - A*x);
\end{lstlisting}

\large{Problem 23 Code}
\begin{lstlisting}[language=MATLAB]
%% CG on the normal equations

close all
clear
clc
n = 500; % A = n x n; b = n x 1
max = 8000; % Max number of iterations
A = randn(n,n);
xt = randn(n,1);
b = A*xt;
x0 = zeros(n,1);
[x_cg,iter,res] = conjgrad(A'*A,A'*b,x0,1e-6);
norm(res(iter))
norm(x_cg-xt)
%% GMRES(500)
close all
clc
x_gmres_500 = gmres(A,b,500,1e-6,max);
norm(x_gmres_500-xt)
%% GMRES(100),i.e., restarted GMRES with m = 100
close all
clc
x_gmres_100 = gmres(A,b,100,1e-6,max);
norm(x_gmres_100-xt)
\end{lstlisting}

\large{Conjugate Gradient for Problem 23}
\begin{lstlisting}[language=MATLAB]
function [x,iter,res] = conjgrad(A,b,x0,tol)
tol2 = tol^2;
x = x0;
r = b - A*x0;
d = r'*r; 
bb = b'*b;
p = r;
iter = 0;
while d > tol2 * bb
do = d;
s = A*p;
alfa = d / (p'*s);
x = x + alfa*p;
r = r - alfa*s;
d = r'*r;
p = r + d/do*p;
iter = iter + 1;
res(iter) = norm(r);
end
end
\end{lstlisting}

\end{document}
