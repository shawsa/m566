\documentclass[12pt]{article}

%%%%%%%%%%%%%%%%%%%%%%%%%%%%%%%%%%%%%%%%%%%%%%%%%%%%%%%%%%%%%%%%%%%%%%%%%%%%%%%%%%%%%%%%%%%%%%%%%%%%
% Math
\usepackage{fancyhdr} 
\usepackage{amsfonts}
\usepackage{amsmath}
\usepackage{amssymb}
\usepackage{amsthm}
\usepackage{enumitem}
%\usepackage{dsfont}

%%%%%%%%%%%%%%%%%%%%%%%%%%%%%%%%%%%%%%%%%%%%%%%%%%%%%%%%%%%%%%%%%%%%%%%%%%%%%%%%%%%%%%%%%%%%%%%%%%%%
% Macros
\usepackage{calc}

%%%%%%%%%%%%%%%%%%%%%%%%%%%%%%%%%%%%%%%%%%%%%%%%%%%%%%%%%%%%%%%%%%%%%%%%%%%%%%%%%%%%%%%%%%%%%%%%%%%%
% Commands and Custom Variables	
\newcommand{\problem}[1]{\hspace{-4 ex} \large \textbf{Problem #1} }
\let\oldemptyset\emptyset
\let\emptyset\varnothing
\newcommand{\norm}[1]{\left\lVert#1\right\rVert}
\newcommand{\sint}{\text{s}\kern-5pt\int}
\newcommand{\powerset}{\mathcal{P}}
\renewenvironment{proof}{\hspace{-4 ex} \emph{Proof}:}{\qed}
\newcommand{\RR}{\mathbb{R}}
\newcommand{\NN}{\mathbb{N}}
\newcommand{\QQ}{\mathbb{Q}}
\newcommand{\ZZ}{\mathbb{Z}}
\newcommand{\CC}{\mathbb{C}}
\renewcommand{\Re}{\operatorname{Re}}
\renewcommand{\Im}{\operatorname{Im}}

\newcommand{\solution}{\vspace{2 ex} \hspace{-5 ex} \emph{Solution.} }


%%%%%%%%%%%%%%%%%%%%%%%%%%%%%%%%%%%%%%%%%%%%%%%%%%%%%%%%%%%%%%%%%%%%%%%%%%%%%%%%%%%%%%%%%%%%%%%%%%%%
%page
\usepackage[margin=1in]{geometry}
\usepackage{setspace}
%\doublespacing
\allowdisplaybreaks
\pagestyle{fancy}
\fancyhf{}
\rhead{Malmuth \& Shaw \space \thepage}
\setlength\parindent{0pt}

%%%%%%%%%%%%%%%%%%%%%%%%%%%%%%%%%%%%%%%%%%%%%%%%%%%%%%%%%%%%%%%%%%%%%%%%%%%%%%%%%%%%%%%%%%%%%%%%%%%%
%Code
\usepackage{listings}
\usepackage{courier}
\lstset{
	language=Python,
	showstringspaces=false,
	formfeed=newpage,
	tabsize=4,
	commentstyle=\itshape,
	basicstyle=\ttfamily,
}

%%%%%%%%%%%%%%%%%%%%%%%%%%%%%%%%%%%%%%%%%%%%%%%%%%%%%%%%%%%%%%%%%%%%%%%%%%%%%%%%%%%%%%%%%%%%%%%%%%%%
%Images
\usepackage{graphicx}
\graphicspath{ {images/} }
\usepackage{float}

%tikz
\usepackage[utf8]{inputenc}
\usepackage{pgfplots}
\usepgfplotslibrary{groupplots}

%%%%%%%%%%%%%%%%%%%%%%%%%%%%%%%%%%%%%%%%%%%%%%%%%%%%%%%%%%%%%%%%%%%%%%%%%%%%%%%%%%%%%%%%%%%%%%%%%%%%
%Hyperlinks
%\usepackage{hyperref}
%\hypersetup{
%	colorlinks=true,
%	linkcolor=blue,
%	filecolor=magenta,      
%	urlcolor=cyan,
%}

\begin{document}
	\thispagestyle{empty}
	
	\begin{flushright}
		Daniel Malmuth \& Sage Shaw \\
		m566 - Spring 2018 \\
		\today
	\end{flushright}
	
{\large \textbf{HW - Chapter 8}}\bigbreak

\problem{1 (a)} Let $P$ be a matrix such that $P^2 = P$ (a projector matrix). Let $\mathbf{x}$ be an eigenvector (non-zero) of $P$ with corresponding eigenvalue $\lambda$. Then
\begin{align*}
	\lambda \mathbf{x} & = P \mathbf{x} \\
	& = P^2 \mathbf{x}\\
	& = P (P \mathbf{x}) \\
	& = P(\lambda \mathbf{x}) \\
	& = \lambda^2 \mathbf{x} \\
	\lambda & = \lambda^2
	0 = \lambda(1- \lambda) \\
\end{align*}
Thus $\lambda = 0$ and $\lambda = 1$ are the only possible eigenvalues of a projector matrix.

\bigbreak
%%%%%%%%%%%%%%%%%%%%%%%%%%%%%%%%%%%%%%%%%%%%%%%%%%%%%%%%%%%%%%%%%%%%%%%%%%%%%%%%%%%%%%%%%%%%%%%%%%%%

\problem{1 (b)} Show that if $P$ is a projector, then $I-P$ is a projector.

\begin{proof}
	Let $P$ be a projector. Then
	\begin{align*}
		(I-P)^2 & = (I-P)(I-P) \\
		& = I^2 - IP - PI + P^2 \\
		& = I - P - P + P \\
		& = I - P
	\end{align*}
	Thus $(I-P)$ is a projector as well.
\end{proof}

\bigbreak
%%%%%%%%%%%%%%%%%%%%%%%%%%%%%%%%%%%%%%%%%%%%%%%%%%%%%%%%%%%%%%%%%%%%%%%%%%%%%%%%%%%%%%%%%%%%%%%%%%%%

\problem{8 (a)} For a matrix $A \in \RR^{m \times n}$ with $m>n$, suppose $A$ is rank $n$ and show that $A^TA$ is non-singular.

\begin{proof}
	Let $U\Sigma V^T = A$ be the singular value decomposition of $A$. Then $U \in \RR^{m \times m}$ and $V \in \RR^{n\times n}$ are orthogonal matrices and $\Sigma \in \RR^{n \times m}$ is rank $n$ and has $n$ non-zero values on its diagonal. Then
	\begin{align*}
		A^TA & = V \Sigma^T U^T U \Sigma V^T \\
		& = V \Sigma^T \Sigma V^T
	\end{align*}
	Since $\Sigma ^T \Sigma$ is an $n \times n$ diagonal matrix with non-zeros on its diagonal it is non-singular. Obviously $V$ and $V^T=V^{-1}$ are non-singular so $A^TA$ is non-singular as well.
\end{proof}

\bigbreak
%%%%%%%%%%%%%%%%%%%%%%%%%%%%%%%%%%%%%%%%%%%%%%%%%%%%%%%%%%%%%%%%%%%%%%%%%%%%%%%%%%%%%%%%%%%%%%%%%%%%

\problem{8 (b)} For a matrix $A \in \RR^{m \times n}$ with $m>n$, suppose $A$ is rank $n$ (full column rank) and show that $A(A^TA)^{-1}A^T$ a symmetric projector, also called an orthogonal projector.

\begin{proof}
	From part $A$ we know that $(A^TA)^{-1}$ exits. Since
	\begin{align*}
		\big( A(A^TA)^{-1}A^T \big)^T & = (A^T)^T \big( (A^TA)^{-1} \big)^T A^T \\
		& = A \big( (A^TA)^{T} \big)^{-1} A^T \\
		& = A \big( (A^T(A^T)^T) \big)^{-1} A^T \\
		& = A(A^TA)^{-1}A^T
	\end{align*}
	we know it is symmetric. Verifying that it is a projector is simply a mater of algebra
	\begin{align*}
		\big( A(A^TA)^{-1}A^T \big)^2 & = A(A^TA)^{-1}A^T A(A^TA)^{-1}A^T \\
		& = A(A^TA)^{-1} (A^T A) (A^TA)^{-1}A^T \\
		& = A(A^TA)^{-1} I A^T \\
		& = A(A^TA)^{-1} A^T \\
	\end{align*}
\end{proof}

\bigbreak
%%%%%%%%%%%%%%%%%%%%%%%%%%%%%%%%%%%%%%%%%%%%%%%%%%%%%%%%%%%%%%%%%%%%%%%%%%%%%%%%%%%%%%%%%%%%%%%%%%%%

\problem{8 (c)} Show that the solution to the linear least squares problem satisfies
$$
\mathbf{r} = \mathbf{b} - A \mathbf{x} = P \mathbf{b}
$$
where $P$ is an orthogonal projector.

\begin{proof}
	We know that the solution to the least squares problem can be expressed in terms of the pseudo-inverse
	$$
		\mathbf{x} = (A^TA)^{-1}A^T \mathbf{b}
	$$
	Then the residual is given by
	\begin{align*}
		\mathbf{r} & = \mathbf{b} - A (A^TA)^{-1}A^T \mathbf{b} \\
		& = \big( I - A (A^TA)^{-1}A^T \big) \mathbf{b}
	\end{align*}
	Let $P = I - A (A^TA)^{-1}A^T$ From parts (a) and (b) we know that this is a projector. Since $I$ is symmetric and the difference of symmetric matrices is also symmetric we know from part (b) that $P$ is symmetric.
\end{proof}

\bigbreak
%%%%%%%%%%%%%%%%%%%%%%%%%%%%%%%%%%%%%%%%%%%%%%%%%%%%%%%%%%%%%%%%%%%%%%%%%%%%%%%%%%%%%%%%%%%%%%%%%%%%

\problem{8 (d)} Express $P$ from part (c) in terms of the QR decomposition of $A$. \bigbreak

Let $A_{m \times n} = Q_{m \times m} R_{m \times n}$ be the QR decomposition of $A$. Then substituting we have
\begin{align*}
	P & = I - A (A^TA)^{-1}A^T \\
	& = I - QR (R^TQ^TQR)^{-1}R^TQ^T \\
	& = QIQ^T - QR (R^TR)^{-1}R^TQ^T \\
	& = Q(I - R (R^TR)^{-1}R^T)Q^T
\end{align*}

\bigbreak
%%%%%%%%%%%%%%%%%%%%%%%%%%%%%%%%%%%%%%%%%%%%%%%%%%%%%%%%%%%%%%%%%%%%%%%%%%%%%%%%%%%%%%%%%%%%%%%%%%%%

\problem{8 (e)} With $\mathbf{r}$ defined as the residual (as usual) let $\hat{\mathbf{b}} = \mathbf{b} - \alpha \mathbf{r}$ and show that the vector $\mathbf{x}$ that minimizes the least squares problem $A\mathbf{x} = \mathbf{b}$ also minimizes the least squares problem $A\mathbf{x} = \hat{\mathbf{b}}$.

\begin{proof}
	Let $\mathbf{x} = (A^TA)^{-1}A^T\mathbf{b}$. Then $\mathbf{x}$ is the solution to the least squares problem for the system $A\mathbf{x} = \mathbf{b}$. Define $\mathbf{r} = \mathbf{b} - A \mathbf{x}$. Let $\alpha$ be a scalar. Define $\hat{\mathbf{b}} = \mathbf{b} - \alpha \mathbf{r}$. The solution to the least squares problem $A\mathbf{z} = \hat{\mathbf{b}}$ is given by
	\begin{align*}
		\mathbf{z} & = (A^TA)^{-1}A^T \hat{\mathbf{b}} \\
		& = (A^TA)^{-1}A^T (\mathbf{b} - \alpha \mathbf{r}) \\
		& = (A^TA)^{-1}A^T \mathbf{b} - \alpha (A^TA)^{-1}A^T \mathbf{r} \\
		& = \mathbf{x} - \alpha (A^TA)^{-1}A^T (\mathbf{b} - A \mathbf{x}) \\
		& = \mathbf{x} - \alpha \big(  (A^TA)^{-1}A^T \mathbf{b} - (A^TA)^{-1}A^T A \mathbf{x} \big) \\
		& = \mathbf{x} - \alpha \big(  \mathbf{x} - (A^TA)^{-1}(A^T A) \mathbf{x} \big) \\
		& = \mathbf{x} - \alpha \big(  \mathbf{x} - I \mathbf{x} \big) \\
		& = \mathbf{x}
	\end{align*}
\end{proof}

\bigbreak
%%%%%%%%%%%%%%%%%%%%%%%%%%%%%%%%%%%%%%%%%%%%%%%%%%%%%%%%%%%%%%%%%%%%%%%%%%%%%%%%%%%%%%%%%%%%%%%%%%%%

{\hspace{-4 ex} \large \textbf{Appendix - Code listings}}\bigbreak



\end{document}
